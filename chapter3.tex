\section{Build tools: Maven}

\subsection{What is it?}

\begin{flushleft}
\textbf{Maven} is basically a tool that can now be used for building and managing any Java-based project. It addresses two aspects of building software: first, it describes how software is built, and second, it describes its dependencies. By using these two aspects, it can be used for a variety of tasks inside a project, which include building executables, building documentation and downloading dependencies. It is also very extensible, so any features that is doesn't contain normally can be added using plugins. 

\end{flushleft}


\subsection{Objectives}

Maven’s primary goal is to allow a developer to comprehend the complete state of a development effort in the shortest period of time. In order to attain this goal there are several areas of concern that Maven attempts to deal with: 

\begin{enumerate}
\item Making the build process easy: provides a lot of shielding from the details.

\item Providing a uniform build system: allows a project to build using its project object model (POM) and a set of plugins that are shared by all projects using Maven.

\item Providing quality project information: provides plenty of useful project information that is in part taken from your POM and in part generated from your project’s sources (change log,dependency list, unit test)

\item Providing guidelines for best practices development: specification, execution, and reporting of unit tests are part of the normal build cycle using Maven.

\item Allowing transparent migration to new features: provides an easy way for Maven clients to update their installations so that they can take advantage of any changes that been made to Maven itself.

\end{enumerate}

\subsection{Components}

\begin{itemize}

\item An XML file describes the software project being built, its dependencies on other external modules and components, the build order, directories, and required plug-ins. 

\item It comes with pre-defined targets for performing certain well-defined tasks such as compilation of code and its packaging. 

\item Maven dynamically downloads Java libraries and Maven plug-ins from one or more repositories such as the Maven 2 Central Repository, and stores them in a local cache.This local cache of downloaded artifacts can also be updated with artifacts created by local projects. Public repositories can also be updated.

\end{itemize}

\subsection{Some Features}

The following are the key features of Maven in a nutshell: 

\begin{itemize}
\item Superior dependency management including automatic updating, dependency closures (also known as transitive dependencies)

\item Able to easily work with multiple projects at the same time.

\item A large and growing repository of libraries and metadata to use out of the box, and arrangements in place with the largest Open Source projects for real-time availability of their latest releases.

\item Extensible, with the ability to easily write plugins in Java or scripting languages.
\item Model based builds: Maven is able to build any number of projects into predefined output types such as a JAR, WAR, or distribution based on metadata about the project, without the need to do any scripting in most cases.

\item Coherent site of project information: Using the same metadata as for the build process, Maven is able to generate a web site or PDF including any documentation you care to add, and adds to that standard reports about the state of development of the project.

\end{itemize}

\subsection{Comparison with Apache Ant}

Please see \textbf{differences} in table \ref{table:differences}

\begin{table}[hb]
    \begin{tabulary}{\textwidth}{cLL}
        \toprule
  \textbf{Aspect} & \textbf{Maven } & \textbf{Apache Ant } \\
        \midrule
  \textbf{Reach } & Covers the whole project build process & Takes care of generating the executable \\
        \midrule
  \textbf{Design} & Uses conventions for the build procedure, and only exceptions need to be written down & All aspects need to be specified \\
        \midrule
  \textbf{languages Supported} & Can also be used to build and manage projects written in C\#, Ruby, Scala, and other languages&Only used for Java applications \\ 
    \bottomrule
    \end{tabulary}
    \caption{Differences}\label{table:differences}
\end{table}

Sources taken from~\autocites{wiki_maven}{apacheorg_maven}

